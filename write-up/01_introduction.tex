\section{Introduction}

% PROMISE
% you are working in the area of X: what promises does X make to the world? why is X exciting enough for people to care?
% The recent development of \xx promises to [make the world a better place] \xxx
Human-Computer Interaction (HCI) is often considered as one of the most interdisciplinary research fields.
% 
% PROBLEM
% but in order to realize X's promise, we have to solve a key problem ...
% However, the problem is \xxx
Thus we hypothesize that HCI papers must have impact beyond HCI, which can be indicated by how often an HCI paper is cited by non-HCI papers.
% , \ie have forward-citation sources across disciplinary.
% 
% PRIOR WORK
% what is the closet related work to your research? how do you differentiate from them? what is the gap in past work?
% To solve this problem, past work has \xxx
Unfortunately, conventional citation metrics, such as h-index or i-10-index, cannot indicate such impact because they only count the number of citations but not the sources of citations.

% PROPOSED SOLUTION
% summarize your research in one sentence
% To fill in this gap, we design and implement \xxx
% use an example (and refer to figure 1) to walkthrough your research in more details
% As shown in \fgref{fig1}, \xxx
To address this, we propose \xin---a simple metric that measures how often a paper's citations cross the disciplinary boundaries.
Given a set of venues representing a specific research field, X-index is defined as the proportion of citations {\it not} coming these venues.

\xac{todo: briefly describe dataset}

\xac{todo: summarize findings}

\xac{todo: show github link}

% PROOF
% what experiments/evaluations/studies have you run to prove that your idea works?
% To validate \xx, we conducted \xxx


% CONTRIBUTION
% {\bf The main contribution} of this paper is \xxx
% differentiate from prior work
% In contrast to prior work that \xxx