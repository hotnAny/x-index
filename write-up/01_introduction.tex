\section{Introduction}

% PROMISE
% you are working in the area of X: what promises does X make to the world? why is X exciting enough for people to care?
% The recent development of \xx promises to [make the world a better place] \xxx
Human-Computer Interaction (HCI) is often considered as one of the most interdisciplinary research fields.
% 
% PROBLEM
% but in order to realize X's promise, we have to solve a key problem ...
% However, the problem is \xxx
Thus we hypothesize that HCI papers must have impact beyond HCI, which can be indicated by how often an HCI paper is cited by non-HCI papers.
% , \ie have forward-citation sources across disciplinary.
% 
% PRIOR WORK
% what is the closet related work to your research? how do you differentiate from them? what is the gap in past work?
% To solve this problem, past work has \xxx
Unfortunately, conventional citation metrics, such as h-index or i-10-index, cannot indicate such impact because they only count the number of citations but not the sources of citations.

% PROPOSED SOLUTION
% summarize your research in one sentence
% To fill in this gap, we design and implement \xxx
% use an example (and refer to figure 1) to walkthrough your research in more details
% As shown in \fgref{fig1}, \xxx
To address this, we propose \xin---a simple metric that measures how often a paper's citations cross the disciplinary boundaries.
Given a set of venues representing a research field, the \xin of the papers in this field is defined as the proportion of citations of these papers {\it not} coming from these venues.

We compiled a list of core HCI venues and analyzed \xin of papers from CHI, UIST, and CSCW (hereafter referred to as {\bf HCI papers}) published between 2010 and 2020\footnote{Our dataset and source code can be found at \url{https://github.com/hotnAny/x-index}.}.
We found that 
\begin{itemize} [leftmargin=0.25in]
    \item \xin of more recently published HCI papers have a lower \xin than earlier papers';
    \item If only considering HCI papers that have existed for at least five years, the more recently-published papers' \xin is still lower;
    \item Amongst all the HCI papers cited in a given year, the earlier papers tend to have a higher \xin than the later ones;
    \item In the more recent years, HCI papers seem less likely to be cited by papers outside of our core HCI venue list.
\end{itemize}




% \xac{todo: briefly describe dataset}

% \xac{todo: summarize findings}

% \xac{todo: show github link}

% \xac{cite the stanford paper}

% PROOF
% what experiments/evaluations/studies have you run to prove that your idea works?
% To validate \xx, we conducted \xxx


% CONTRIBUTION
% {\bf The main contribution} of this paper is \xxx
% differentiate from prior work
% In contrast to prior work that \xxx