\section{Discussions}

\subsection{Limitations of the Analyses}

First, \xin intends to indicate how much non-HCI papers cite HCI papers, yet our classification of non-HCI papers is based on a list of core HCI venues. Thus some non-HCI citations might have come from HCI venues not included in our list. A common example of this issue is various HCI workshops over the years.
As a consequence, the resultant \xin in our analyses is actually inflated---the real numbers should be lower after removing the `noisy' HCI venues from the currently-considered non-HCI citations.

Second, as mentioned earlier, we rely on Lens.org API \cite{TheLensF23:online} that does not seem to retrieve all citations. 
We do not believe such a limitation invalidates our analyses unless this API acts biasedly, \ie intentionally misses either HCI or non-HCI citations.
Future work can union multiple tools and sources (\eg Google Scholar, Semantic Scholar, and Microsoft Academic) to approximate a full coverage of citations.

Third, \xin is just a number and it would be great to perform more detailed analyses, such as breaking down the citation sources of CHI.
Future work can employ a more intelligent method than our simple keyword-matching approach to dissect the disciplinarity of citation sources (\eg $X\%$ HCI, $Y\%$ Computer Vision, and $Z\%$ Psychology).

\subsection{Interpreting the Results}

Does our \xin analyses mean that HCI has a decreasing impact across the disciplinary boundary?
It seems so; at least we can say that relatively fewer and fewer papers outside of the core HCI venues are citing HCI papers.

One counter-argument against the above could be that HCI venues themselves might have become more and more interdisciplinary. 
Thus citations coming from within these HCI venues can also indicate impact across what used to be the disciplinary boundary.

It is also possible that the number of HCI venues is growing much faster that of non-HCI venues.
In that case, even if more and more non-HCI venues are citing HCI papers, \xin will still decrease because the denominator is increasing faster than the nominator.

It will be interesting for future work to analyze \xin of other research fields. 
For example, according to Google Scholar, as of March 2023, CVPR has the highest h-index amongst all ``Engineering \& Computer Science'' publications---Should we expect CVPR to have a high \xin?

Finally, regardless of how we should interpret these analyses, as a field, HCI should develop more awareness of its impact beyond HCI (\eg a recent analysis by Cao \etal on industrial impact \cite{cao_breaking_2023}), rather than remaining complacent with validating each other's work internally via having more and more HCI papers cite other HCI papers.
